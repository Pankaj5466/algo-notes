\chapter{Backtracking}\label{chp:backtracking}

This is one of the fundamentals of algorithm design.
Backtracking and recursion are closely related.

\faQuestion \hspace{5mm} When do we need backtracking?\\
Hint: We need backtracking when we need to try to all permutation!

For example following scenerio require backtracking:
\begin{compactenum}[(i)]
    \item Arrange k balls of different color on x axis. Print all arrangement.
    \item Print all the ways k alphabet can be arranged when each character is uniqueu.
    \item print all the string you can generate with given string s. In case when duplicate are allowed as string character.
    \item In how many many ways you can place 2 queen on n*n chessboard.
    \item What is the maximum number of queen (bishop,knight,etc) you can place on N*N chessboard.
    \item Generate Parenthesis when you can use at most k '(' and ')'.
    \item Subset sum problem.
\end{compactenum}

% All of these question can be solved with backtracking. The first algorithm you should try to solve the given problem is backtracking, then recursion.
% If the computed time complexity is not within range, then you can memoize the solution.

\medskip
\begin{fullwidth}
\begin{code2}
- Brute Force approach finds all the possible solutions and selects desired solution per given the constraints.
- Dynamic Programming also uses Brute Force approach to find the OPTIMUM solution, either maximum or minimum. (DP solves the count problem)
- Backtracking also uses Brute Force approach but to find ALL the solutions.

- Solutions to the Backtracking problems can be represented as State-Space Tree.
- The constrained applied to find the solution is called Bounding function.
- Backtracking follows Depth-First Search method.
- Branch and Bound is also a Brute Force approach, which uses Breadth-First Search method.
\end{code2}
\end{fullwidth}


Keywords:
state space tree
DFS(Backtracking),BFS(Branch and BOund)

\begin{exercise}[Backtracking Theory Questions]
    \begin{compactenum}
        \item Print all the subset of given array.
    \end{compactenum}
    
\end{exercise}


\begin{problem}[Subset Generation | $O(2^n)$]
    ok
\end{problem}