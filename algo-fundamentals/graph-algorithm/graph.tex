\chapter{Graph Algorithm}\label{ch:graph}

New data-strucuture that will lead us to algorithm on graphs.

The reason why graph algorithm are cool behides having tons and tons of applications is because they force you from a programmers point of view to really distinguish between some intutive idea you have about how to do something and actually making that implementation work.

A algorithm is about two things, clear ideads and getting that intutive idea out of you and describing it, but once you've described it there's  still a way to go and defining that data-structure and make it work.

\section{Graph Representation}

\begin{lfigure}{example-image}{0.5}{0.5}
    \lipsum[3]
\end{lfigure}

\codecaption{Adjancency List}
\begin{lfigure}
    {example-image}{0.5}{0.5}
    
\end{lfigure}

There are general techniquie on how to traverse the graph, namely depth-first-search(dfs) and breadth-first-search(bfs).

DFS and BFS are basic tools that becomes the building block of other fancy graph algorithm.\\
ex: finding  a cycle can be done using a variant of dfs.

A cool example using dfs tool is finding strongely disconnected component in class.

Other graph algorihtms:
\begin{exercise}[Graph Algorithm Summarization]
    \textbf{Famous Graph Algorithms:}
    \begin{compactenum}
        \item  Shortest path
        \item Minium Spanning tree
    \end{compactenum}

    \textbf{Non Famous Algorithms:}
    \begin{compactenum}
        \item Given a graph, can you tell if its a plainer graph or not.
        
        A graph is plainer graph, if you could draw it without crossing any edges. (its importatnt because, some algorithm run faster on plainer graph)
    \end{compactenum}
\end{exercise}

\newpage
\section*{Topological Sorting}
A topological sort of graph make sense only when the graph is directed.

\begin{marginfigure}
    \raggedright
\textbf{algorithm and data-structure tradeoff.}

There is always a trade-off between data strucutre and algorithms.
\vspace*{2mm}

The more fancy things you keep in your data-structure usually the less work your algorithm has to do. Conversly, the less fancy things you keep in your data structure, the more work your algorihtm has to do.
\end{marginfigure}

 For topological sorting, we will first try to solve it without storign anything, then notice that its works a little than we like. Then later when we make our data-strucutre fancier and do a extra work in beginning, then the work we do later is little less.

\begin{marginfigure}
    \vspace{2mm}
    
    \raggedright
    \textbf{
        Never think of a algorithm without a data-structure}.
\end{marginfigure}

\vspace{5cm}
\begin{lfigure}{example-image}{0.5}{0.47}
    You can visualize each node as course of a university, and to complete a course you must complete all course before it.

    Basic Idea:
    \begin{compactenum}
        \item Find all nodes that have no arrow going into them. (indegree)
        \item Delete it. Output it.
        \item Got to step 1 till graph is not empty.
    \end{compactenum}
\end{lfigure}

Will the above algorithm works always? what if you cannot fund a node with 0 indegree but the graph is not dempty.

Above case will only happen if there is cycle is a graph. In fact this is a good way to detect cycle in a direct graph.(though it wont tell you what is the cycle.)

\textbf{Topological sorting only works on DAG.} If the graph has cycle, then Topological orderting has no meaning for that graph.

\medskip
\begin{code3}[Topological Sorting]
    Topological Sorting Code Here
\end{code3}

\section{Minimum Spanning Tree}\label{ch:minimum-spanning-tree}

Minimum spanning tree is a really  baisc graph algorithm, it requires a little more than the brute-force algorithm that we just applied on topolocial sorting.

It serves as a fine example on how data-strucutre and algorithm go altogether.

There are two algorithm to solve this, and they are \textbf{Prim} and \textbf{Kruskal}.

\codecaption{Prims Algorithm | DS: Heap+Graph}
Prim's algorithm create the minimum-spanning-tree in incremental manner. It select a node which has minimum cost and include that in growing tree forest.

\begin{lfigure}{resources/topological.png}{0.3}{0.67}
    Let,\\
    \verb|parent[v]:= parent node of vertex v in minimum spanning tree|
    \verb|cost[v]:= cost to pay if we want to include vertex v in minimum spanning set.|

    Before start, set \verb|parent[u]=-1, cost[u]=INF|

    \begin{compactenum}
        \item set cost[0] = 0
        \item Pick a node from heap that has lowest cost. Let this node be u.
        \item Process this node. Processing means, update the cost[v] and parent[v]. where \verb| u-(w)>v| represent edge from u to v with weight w.
        \item Delete this node and go to step2 again. Repeate this until there is no node in heap. 
    \end{compactenum}

\end{lfigure}

\codecaption{Kruskal Algorithm | DS: union-find+Graph}
\begin{lfigure}{resources/Kruskal.png}{0.3}{0.67}

    In Kruskal's agorithm we select the edge which has minimum weight and include it in tree forest. If the selected edge causes to form a cycle, then we discard such edge and continue processing rest of the edge.

    Kruskal Algorithm usages union-find data-structure to detect if including the edge will result in cycle or not.
\end{lfigure}

Disjoint Union Find with path compression (union-find) is really good example to show  how amortized complexity works.

\section{Graph Traversal}

Both BFS and DFS go through the graph and process each node once. But they do it in different order.
The question is what they are going to do when they go through the node.

Sometimes they can do a very complexity things, which result in different algorithms. All graph algorithm usages bfs/dfs and do different task while going through node, which result in different algorithm.

The data strucutre that go with BFS is queue, the data strucutre that go with DFS is stack.

\marginnote{DFS has much more use than BFS due to its recursive structure.}

\rfl{The key thing about DFS is that, it start going down first, so when it backs its way up, it go so much information, that  the node which has called it can use this data to process node and make decision.}

\subsection{DFS}
Terminology associated with dfs:
\begin{compactenum}
    \item tree edge, back edge, forward edge, cross edge
    \item start time, finish time,
    \item parent[] to keep track of tree structure.
\end{compactenum}
marked edge are tree edge.
ED: is cross edge (edge that go from one end of tree to another end of tree.)

If there be edge EA than that would a back edge. (back edge exist if there is cycle in the graph.) Back edge goes backward in the tree, forward edge goes forward in the depth-first-tree.

For a undirected graph, we have only tree edge and back edge.
Distinguistion betwwen tree edge and back edge is of most important.

\codecaption{DFS Traversal}
\begin{lfigure}{resources/dfs.png}{0.35}{0.62}
    \textbf{Types of tree edge:}
   \begin{compactenum}
        \item Highlighted edge are \textbf{tree-edge}. They are the edge which form the dfs-forest.
        \item Edge FE,ED etc are \textbf{cross-edge}.
        \item Edge EA is a \textbf{back-edge}. (this graph does not have this edge)
        \item GH is \textbf{forward-edge}. It connect to a node forward in the dfs-tree.
   \end{compactenum}

   The time at which the node is first visited is called \textbf{start time}.

   The time at which the exploration of node is completed is called \textbf{finish time}.

   Any thing that can be done with start time, can be done via BFS.
\end{lfigure}

\codecaption{DFS Code Skeleton}

\begin{lfigure}{resources/dfs-skeleton.png}{0.35}{0.62}
   This is the bare minimum code. With graph algorithm you do not re-invent the wheel, instead you use this dfs-skeleton and put on material at possible places as per your requirement.
   
   There a lot of places to put code, some of which are:
   \begin{compactenum}
        \item During visited marking.
        \item During child visit.
        \item At the end of DFS stack.
        \item You can also make dfs return some value and use it in code. (ex: sum of child below current node. or in DP)
   \end{compactenum}
\end{lfigure}

\codecaption{DFS Code in Depth}
\begin{fullwidth}
    % \verbatiminput{code/graph-traversal.cpp}
    \lstinputlisting[frame=none,breaklines=true,basicstyle=\small\ttfamily]{code/graph-traversal.cpp}

\end{fullwidth}

Applications of DFS:
\begin{compactenum}[(i)]
\item Topological sort and finishting time relationship. (finishing time in revese will give us topolocial sort of the given graph.)
\item Graph Coloring Problems
\item Find start time and finish time of visited.
\item Detect Cycle in Directed Graph
\item Detect Cycle is Undirected Graph. (basicall add extra check to check that v is not par[u])
\item Find all strongely connected component of a given Graph.(there is no such thing as strongely connected component of undirected graph.)
\end{compactenum}

\codecaption{Topological Sorting}
\begin{lfigure}{example-image}{0.35}{0.62}
    Find out the finish time of each node. Then when you print out nodes in the reverse order of their finish time, that is the topological ordering of the graph.

    You could have just a stack, and push node into the stack where you are updating the finish time of all node. This way you won't have to process the end time also.
\end{lfigure}


\codecaption{Strongely Connected Component}
Defination: a set of node in a graph, for which you can reach any other node in the group are called strongely connected.

A analogy of its application,is to think that as a miltary scientist, you want to destory you enemy road network. Now you have to find out as which points to destroy to affect their network in worst way.

\begin{lfigure}{resources/strongely-connected-component.png}{0.35}{0.62}
   \textbf{Algorithm To Find All Strongley Connected Component}
   \begin{compactenum}
    \item DFS - Find the finishing time of each vertex.
    \item Reverse the edges in the graph.
    \item Call DFS on the node in reverse graph, in reverse order of their finish time.
   \end{compactenum}

   Its hard to describe how it works, and its very high level and involves a lot of mathematics.
\end{lfigure}

