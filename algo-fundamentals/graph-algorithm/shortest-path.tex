\section{Shortest Path (TC: greedy)}
% TC is Alogrithm Techniqueue
\marginnote{An anology of -ve weight on graph can be pusing water up the hill.}

\marginnote{Finding shortest from single source to single target, also finds out shortest path to all other nodes. And nobody is able to devise a algorithm that does better for first case only.}[2cm]

There are a lot of versiion of shortest path. \rfl{They all scan the vertex and update the distance[] and parent[], the main differance between any of them is the \textbf{order in which the scanning of vertex is done}}.
\begin{compactenum}
    \item graph is undirected
    \item graph is directed.
    \item All +ve Edges (aka Dijkastra Algorithm, works with directed/undirected)\\$O(n^2)$ with matrix representatio. and $O(nlog(n))$ with Adjacency List with Heap.
    \item On DAG. (algo. Breadth First Scanning and Bellman Ford) (directed graph but no cycle) (even with -ve edge weight) | $O(e)$
    \item General Case. (-ve weight and can have cycle) (what we are not allowing is that a cycle that has -ve weigh.) |a polynomial time algoritm | $O(n*e)$
    
    Having a -ve weight cycle, means you can have -INF shortest path, which has no pratical meaning.

    \item with -ve weight, and can have -ve weight cycle, but you are not allowed to use such cycle. | a NP Complete Problem.
    
\end{compactenum}

\rule{\linewidth}{0.2em}
\rfl{Shortest path works on Greedy Technique}
\intution{we are gonna keep track of current distance[] and parent[] of the shortest path tree. And we are going to update as we move along. We are gonna in a greedy way by taking local minimuma, if later we found out a new local minimuma for same node, we will scan that node again.  When we are all done with local minima, then we will have our overall best shortest path.}

Order of scan for DAG?\\
If you scan the nodes in topological order, then you are sure that a given node never me scanned twice!

Longest Path vs Shortest Path.\\
Change the edge sign and find shortest path of the DAG.

Longest path in a tree?\\
This is well known problem, known as Diameter of a tree.

\rfl{A tree is a special graph, that does not have a cycle. It means a tree does not have any back edge, and we don't talk about shortest path in tree, we talk about only one path.}

Diameter of a graph?\\
Take any two node of the graph, find the shortet path between them. Then the longest of those will be the diameter.(A graph can have more than one path to other node, while tree has always unique path)
\\This is longest we have ever have to traverl from going one node to other.

Breath First Scanning: Scan node in their BFS order.

\rule{\linewidth}{0.2em}

\codecaption{Dijkastra}
\intution{The constrain of only +ve edges, means the graph cannot have -ve weight for edge or a cycle.}
\begin{lfigure}{resources/dijkastra.png}{0.3}{0.65}
    Dijkastra solves shortest-path problem when we have only \rfl{+ve edges} in the graph. \rfl{it works for directed as well as undirected graph}

    We keep all the node in the heap, then at each iteration we take minimum dis[] node from the heap and scan it.

    Theorem x.y.z ensures that, if we select minimum from the heap to scan at each time, then each node is scanned \textbf{only once}. ie. you scan every node exactly once. Once we pop the element from heap/red-black-tree, then you can be sure that you will no relable it again.

    \textbf{Complexity}\\
    Complexity of whole scan = (No of edges)
    at each step we select minimum, whose compelxity is log.
    Hence, overall complexity = $O(ElogV)$

    \footnotetext{\href{https://www.youtube.com/watch?v=DiedsPsMKXc}{theory-ref}}

    \footnotetext{\href{https://cp-algorithms.com/graph/dijkstra_sparse.html}{code-ref}}
\end{lfigure}

\codecaption{Dijkastra Implementation}
\begin{fullwidth}
    \begin{minipage}{0.5\fullfigurelen}
        \begin{code3}[Dijkastra With set]
const int INF = 1000000000;
vector<vector<pair<int, int>>> adj;
vector<int> d; //dis[]
vector<int> p; //par[]

void dijkstra(int s) {
    int n = adj.size();
    d.assign(n, INF); 
    p.assign(n, -1); 

    d[s] = 0;
    set<pair<int, int>> q;
    q.insert({0, s});
    while (!q.empty()) {
        int u = q.begin()->second;
        q.erase(q.begin());

        for (auto [v,wt] : adj[u]) {
            if (d[u] + wt < d[v]) {
                q.erase({d[v], v});
                d[v] = d[u] + wt;
                p[v] = u;
                q.insert({d[v], v});
            }
        }
    }
}

        \end{code3}
    \end{minipage}
    \begin{minipage}{0.5\fullfigurelen}
        \begin{code3}[Dijkastra with pq]
void dijkstra(int s) {
    int n = adj.size();
    d.assign(n, INF);
    p.assign(n, -1);

    d[s] = 0;
    using pii = pair<int, int>;
    priority_queue<pii, vector<pii>, greater<pii>> q;
    q.push({0, s});
    while (!q.empty()) {
        int u = q.top().second;
        int d_u = q.top().first;
        q.pop();

        /*ignore invalid heap entry*/
        if (d_u != d[u]) 
            continue;

        for (auto [v,wt] : adj[u]) {
            if (d[u] + wt < d[v]) {
                d[v] = d[u] + wt;
                p[v] = u;
                q.push({d[v], v});
            }
        }
    }
}
        \end{code3}
    \end{minipage}
\end{fullwidth}

\newpage
\codecaption{Topological Scanning | O(e)}
For shortest path for a topological graph, if we just scan the vertex in their topological ordering, then after all the scan , we will our shortest path answer.

\codecaption{Breadth First Scan | O(en)}
Finding shortest-path with -ve edge weight. (But no -ve weight for cycle. i.e graph can have cycle with +ve weight)
\begin{lfigure}{resources/bredth-first-scan-sp.png}
{0.4}{0.57}  

The order of scanning in breadth-first-scan is BFS. Any time we found that a already visited node can be reached in less distance, we \textbf{relable} that node and add it to queue again.

Thre is a limit to the number of phases in the queue you can have for this algorithm. Its equal to V-1. (theorem x.y.z)
\end{lfigure}

\textbf{Justification as why number of phase has limit of V-1}\\
Plausible reason as why we have this many phases in this graph. Here's is the plausible reason.

In Dijkastra algorithm, once we scan something, you know it's got the best distance. In Toplogical Sorting, once you scan something, you know it has got the best distance. But, \textbf{in this algorithm after phase 0, all path that have zero edge have their correct distance}, after phase 1, all path that have one edge have their correct distance. After phase 2, all path that have 2 edge have their correct distance.

Now since any two node can have a maximum of V-1 edge between them. Hence, we need to run above algorithm for V-1 phase.

\rfl{Observe that complexiyt $O(en)$ is much slower that bfs/dfs complexity which usually is $O(e+n)$} The reason is quite simple, in bfs/dfs each node is visited exactly once, but in here it can be visited again if it got re-labeled.

\codecaption{Bellmand Ford Algorithm}
\textbf{Bellman Ford algorithm} is a chaotic version of above algorith. \footnote{\href{https://youtu.be/XIAQRlNkJAw?t=2335}{theory-ref}} 
Bellman-Ford says that: go through every edge of your graph and scan forward through those edges. When you are all done do it again and again till n times. When you are all done, you will have shortest-path from source to each vertex and in can proff it to you.

\textbf{breadth-first-scan} says, don't scan everything! Scan that got re-labeled only. And i can argue that,you don't have to scan more that n times.




\codecaption{All pair Shortest Path (TC:dp)}
Start solving the problem from scratch.
\intution{sometimes its easier to solve a problem if you make it harder.}

We will discuss algorithm that will evolve int the following manner:
$O(n^4) \rightarrow O(n^3*log n) \rightarrow O(n^3) $

Let \verb|f(x,y,k,_):=| shortest path to get from node-x to node-y when you are allowed to use at most k edge.

With above defination, the recurrance relation is: \begin{align}
    f(x,y,k) &= \sum_{m=1}^{m=n} f(x,m,k-1) + f(m,y,1)\\
    &=\sum f(x,m,k-1) + weight(m,y)
\end{align}

\begin{code3}[code]
    //Not fully tested
    //Base Case
    for(int x=1;x<=n;x++)
    {
        for(int y=1;y<=n;y++)
        {
            dp[x][y][1] = g[x][y]; //adj matrix graph
        }
    }
    
    //Main Algorithm
    for(int k=2;k<=n-1;k++)
    {
        for(int x=1;x<=n;x++)
        {
            for(int y=1;y<=n;y++)
            {
                for(int m=1;m<=n;m++)
                {
                    dp[x][y][k] = 
                    min(dp[x][y][k], dp[x][m][k-1]+dp[m][y][1])
                }
            }
        }
    }
    //complexity: O(n^4)
\end{code3}

\codecaption{Floyd Warshell all pair Shortest Path (TC:dp)}
As of all DP problem, the main enginnering is involved with the defination of the recurrance relations.

Let, \verb|f(x,y,k):=| shortest path from node-x to node-y when we are allowed to use node 1...k

Now at k, we have option to either include the node-k or do not include it. Keeping this dp-intution in mind, our recurrance relation can be written as:
\begin{align}
    val1 &= f(x,y,k-1)\\
    val2 &= f(x,m,k-1) + f(m,y,k-1)\\
    f(x,y,k) &=  \min(val1,val2) \forall m\in[1,n]
\end{align}

\begin{fullwidth}
    
\begin{minipage}{0.5\fullfigurelen}
\begin{lstlisting}[style=scode,caption=Recursive Code]
//code not tested yet
int findAns(int x,int y,int k)
{
    if(k==0)
        return g[x][y]; //weight(x->y)

    int val1; //do not use k^th node
    int val2; //use the k^th node

    val1 = findAns(x,y,k-1);

    //k^th node can be any one amoung n
    for(int m=1;m<=n;m++) 
    {
        val2 = 
    min(val2,findAns(x,m,k-1) + findAns(m,y,k-1));
    }

    return min(val1,val2);
}

int main()
{
    for(int x=1;x<=n;x++)
    {
        for(int y=1;y<=n;y++)
        {
            findAns(x,y,n-1);
        }
    }
}
//Amortized Complexity: O(n^3)
\end{lstlisting}
\end{minipage}
\begin{minipage}{0.5\fullfigurelen}
\begin{code3}[Iterative Code]

    //base case
    for(int x=1;x<=n;x++)
    {
        for(int y=1;y<=n;y++)
        {
            dp[x][y] = g[x][y]; //weight(x->y)
        }
    }
    //main logic
    for(int k=1;k<=n-1;k++)
    {
        for(int x=1;x<=n;x++)
        {
            for(int y=1;y<=n;y++)
            {
                dp[x][y] = min(dp[x][y],
                            dp[x][k]+dp[k][y]);
            }
        }
    }
\end{code3}
\end{minipage}
\end{fullwidth}
