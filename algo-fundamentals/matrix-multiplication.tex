\begin{problem}{Matrix Chain Multiplication}
    Given a sequence of matrices, find the most efficient way to multiply these matrices together. The efficient way is the one that involves the least number of multiplications.
    The dimensions of the matrices are given in an array arr[] of size N (such that N = number of matrices + 1) where the ith matrix has the dimensions (arr[i-1] x arr[i]).
    \href{https://practice.geeksforgeeks.org/problems/matrix-chain-multiplication0303/1}{Pratice Link}
\end{problem}

\begin{solution}[Discuss]
    let $f(idx,_):= $ multiplication cost when we are allowed to multipliy arr[0...idx] \textbf{only}.
    Now at index idx, we want to include the arr[idx] in our solution range \& we already know answer of arr[0...idx-1].

    Hence, at idx we have the following option:
    \textit{can you think of recursive way as, how can we find the solution of $f(idx)$ using subproblem??}

    One way is that we try every possible split: 
    split at 1, split at 2, split at 3 

    % TO-DO:show below code in time
    $ val_k =( leftCost * arr[k] * rightCost )$

    
    $ val_k = f(k)*arr[k]* f(k...idx) $
    
    As you can see for rightCost calculation we need range 
    But our function can only give $f(0..idx)$.  which we usually shorthand for $f(idx)$.

    if you could devise a way to find rightCost, then we can find the answer with above function signature.


\end{solution}

\vspace{5mm}
\begin{solution}[Range Split | $O(n^2)$]
    
    \obeylines
    As we see $f(0..idx)$ (also represented a $f(idx)$) was not sufficent to able to find our answer.
    and we needed a range of l \& r.

    % \vspace{2mm}
    \medskip
    Lets follow up on that, and define our updated recurrance relation as:
    \textbf{$f(l,r,\_):=$ multiplication cost when we are allowed to multiply in range arr[l..r] \textbf{only}.}

    \bigskip
    Now expressing $f(l,r)$ in term of subproblem, we have:
    \marginnote{q: In how many ways we can multiply matrix f(l,r)?}

    For above we have following option:
        try splitting at all possible points.
        i.e split at 1, split at 2, split at 3 ,\dots

        Hence,
        $val_k = leftCost * arr[k] * rightCost$

        $val_k = f(0,k-1) * arr[k] * f(k+1,r)$

        We will take the best of all.

\begin{verbatim}
        int matrixChainMultiplicatino(int l, int r, const vector<int>&arr)
        {
            if(l>=r)
                return 0; //no element exist in arrr

            int bCost = INT_MAX; //best cost

            //Try split at all possible place
            for(int k=l;k<r;k++)
            {
                int leftCost =  matrixChainMultiplicatino(l,k,arr);
                int rightCost = matrixChainMultiplicatino(k+1,r,arr);

                cc printf("(%d,%d):: LeftCost = %d, RightCost = %d\n",l,r,leftCost,rightCost);
                int splittingCost = (arr[l-1] * arr[k] * arr[r]) + leftCost + rightCost;

                bCost = min(bCost,splittingCost);
            }
            return bCost;
        }
\end{verbatim}

\end{solution}