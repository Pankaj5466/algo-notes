\begin{problem}{kmp prefix function computation}

    Applications:
    \begin{asparaenum}
        \item Check the the give string pat exist in given string s or not.
        \item Count all the occurrance of it.
        \item Count the number of occurrance of each prefix. (TO-DO:)
        \item TO-DO: list them all down
    \end{asparaenum}
    \footnotetext{Pratice Link: \href{https://cp-algorithms.com/string/prefix-function.html}{ref}}
\end{problem}



\begin{solution}[BruteForce | $O(n^3), O(n^2)$]

    prefix function visuallization.\\
    prefix function $O(n^3)$ computation.\\
    prfixs function $O(n^2)$ computatio.

    \codecaption{Understanding prefix function structure}
    \begin{lfigure}{./resources/kmp-1.jpg}{0.80}{.15}
        prefix function visuallization.\\
        prefix function $O(n^3)$ computation.\\
        prfixs function $O(n^2)$ computatio.

    \end{lfigure}
\end{solution}

\begin{solution}[Linear Time Algorithm | $O(n)$]

    \intution{past computed $\pi[i]$ have iformation of already matched index. We can use it to skip re-matching of character second time.}

    \codecaption{KMP Algorithm Recursive Match}
    \begin{lfigure}{./resources/kmp-2.jpg}{0.80}{.15}
       $j = \pi[j-1]$ visuallization
    \end{lfigure}


\begin{code3}[prefix function computation in linear time]
vector<int> prefix_function(string s) {
    int n = (int)s.length();
    vector<int> pi(n);
    for (int i = 1; i < n; i++) {
        int j = pi[i-1];
        while (j > 0 && s[i] != s[j])
            j = pi[j-1];
        if (s[i] == s[j])
            j++;
        pi[i] = j;
    }
    return pi;
}
    \end{code3}
\end{solution}